%! Author = Javier Mérida
%! Date = 2/12/2023

% Preamble
\documentclass[12pt]{report}
\parskip=\baselineskip
\setcounter{secnumdepth}{4}
% Packages
\usepackage[utf8]{inputenc}
\usepackage{graphicx}
\usepackage[smartEllipses]{markdown}
\usepackage{setspace}
\usepackage[table,xcdraw]{xcolor}
\usepackage{float}
\singlespacing
% Graphics
\graphicspath{ {images/} }
\floatplacement{figure}{H}
% Verbatim
\usepackage{framed,color,verbatim}
\definecolor{shadecolor}{rgb}{.9, .9, .9}
\newenvironment{code}%
{\snugshade\verbatim}%
{\endverbatim\endsnugshade}
% title
\title{
        {Static Site Generator}\\
    {\large La Salle Campus Barcelona}\\
    {\includegraphics{lasalle-logo}}
}
\author{Javier Mérida}
\date{12 02 2023}

% Document
\begin{document}

    \maketitle


    \chapter{Trying out popular SSGs}\label{ch:trying-out-popular-ssgs}

    The following SSGs has been chosen based on their popularity in order
    to be tried to gather information on what the basic features for SSG
    are.
    It's important to notice that these are being tried in a Linux environment
    (Linux inside Windows, thanks to WSL), installing them from scratch, using vim
    as text editor to minimize external tools assistance to have an unbiased appreciation
    of each one of them.

    \begin{itemize}
        \item NextJS (https://nextjs.org/)
        \item Hugo
        \item VuePress
        \item Eleventy
        \item Astro
    \end{itemize}

    % Trying NextJS
    \markdownInput{chapters/testing-ssgs/nextjs.md}

    % Trying Hugo
    \markdownInput{chapters/testing-ssgs/hugo.md}

    % Trying VuePress
    \markdownInput{chapters/testing-ssgs/vuepress.md}

    % Trying Eleventy
    \markdownInput{chapters/testing-ssgs/eleventy.md}

    % Trying Astro
    \markdownInput{chapters/testing-ssgs/astro.md}


    \section{Comparison Table}\label{sec:comparison-table}


    The following table outlines the features that has been considered as most important to take into account when
    considering static site generator.
    Note that the purpose of this document is not to benchmark any of these features,
    thus there is not a workload test to compare the performance of the explored SSGs. Instead, this is a simple
    exploration on what features are stand out in order to consider as requisites to build a new SSG.


    Also, it is important to note that the lack or the presence of any of these characteristic do not make any SSG better
    than others, as each one of them are made for a specific purpose and shine in terms of addressing the obstacle
    they are meant to.
    In fact, this section must be taken as an exploration of requisites to be taken into account.


    % Please add the following required packages to your document preamble:
% \usepackage[table,xcdraw]{xcolor}
% If you use beamer only pass "xcolor=table" option, i.e. \documentclass[xcolor=table]{beamer}
    \begin{table}[H]
        \begin{tabular}{|
                >{\columncolor[HTML]{EFEFEF}}c |c|c|c|c|c|}
            \hline
            \cellcolor[HTML]{C0C0C0}      & \cellcolor[HTML]{EFEFEF}NextJS & \cellcolor[HTML]{EFEFEF}Hugo & \cellcolor[HTML]{EFEFEF}VuePress & \cellcolor[HTML]{EFEFEF}Eleventy & \cellcolor[HTML]{EFEFEF}Astro \\ \hline
            \textbf{Easy to use}          & Yes                            & Yes                          & No                               & Yes                              & Yes                           \\ \hline
            \textbf{Flexible}             & No                             & No                           & No                               & Yes                              & Yes                           \\ \hline
            \textbf{Strict structure}     & No                             & Yes                          & No                               & No                               & No                            \\ \hline
            \textbf{Template system}      & No                             & Yes                          & No                               & Yes                              & Yes                           \\ \hline
            \textbf{Themes}               & No                             & Yes                          & No                               & Yes                              & Yes                           \\ \hline
            \textbf{Performance oriented} & Yes                            & Yes                          & NA                               & Yes                              & Yes                           \\ \hline
            \textbf{Framework dependant}  & Yes                            & No                           & Yes                              & No                               & No                            \\ \hline
            \textbf{Easy to install}      & Yes*                           & Yes                          & Yes*                             & Yes*                             & Yes*                          \\ \hline
            \textbf{File-based routing}   & Yes                            & Yes                          & NA                               & Yes                              & Yes                           \\ \hline
        \end{tabular}\label{tab:table}
    \end{table}


    Note*: The installation difficulty is strickly dependant of NodeJS and how `npm` addresses any missing dependency,
    which can be (sometimes) a painful issue to solve.

    % SRS


    \chapter{Software requirements specifications}\label{ch:software-requirements-specifications}

    % SRS Introduction.
    \markdownInput{chapters/requirements/introduction.md}

    % SRS Overall description
    \markdownInput{chapters/requirements/description.md}

    % SRS Specific requirements.
    \markdownInput{chapters/requirements/requirements.md}

    \setkeys{Gin}{width=1\linewidth}
    % Software design
    \markdownInput{chapters/software-design/design.md}

    % Used technologies
    \markdownInput{chapters/used-tech/technologies.md}

    \setkeys{Gin}{width=0.7\linewidth}
    % Implementation
    \markdownInput{chapters/implementation/implementation.md}

    % Results
    %! Author = javif
%! Date = 9/11/2023

% Preamble
\documentclass[11pt]{article}

% Packages
\usepackage{amsmath}

% Document
\begin{document}



\end{document}

    % Conclusions
    %! Author = javif
%! Date = 9/12/2023


\chapter{Conclusion}\label{ch:conclusion}

As demonstrated on previous chapter, VaGo is a software capable of obtaining input markdown files from an established
folder, read each file content, navigate through the abstract syntax tree, obtain the required tokens to build up a
web page, parse it into respective HTML tags, create HTML files accordingly to the input content using the same name,
build up dedicated styles using a theme system reading customization variables, read configuration file to adapt
system to user specifications, provide a listener server on a provided port, route requests to map the same file name
convention to target URLs, display insightful logging and an intuitive command interface with discerning usage
documentation.


As a result, it has been shown that the implementation of VaGo successfully meets the requirements outlined in
previous sections.
Furthermore, it can be regarded as a favorable alternative to existing Static Site Generators, as
it possesses the ability to execute a majority of the characteristics commonly observed in the present market.
This is without accounting for some other measures that are not the primary emphasis of this project, such as its
lightning-quick creation and serving of files, which opens up new possibilities beyond merely serving static
information: The rapid pace of building construction facilitates the expansion of serverside rendering capabilities.


What's more, VaGo has been developed using a minimalistic approach, reducing the code and the cognitive load to the
lowest possible, with a balance between fulfilling requirements and following the proposed design with the desired
simplicity.
This minimalistic philosophy has demonstrated an overall increased productivity in terms of implementation, while
reducing the amount of complications and challenges to face when dealing with bugs and errors, with the addition of
being an open door to extension given that the code base is very simple and easy to read and understand, allowing
future collaborators to improve its functioning and/or add new features.

All in all, VaGo has been capable of meeting the requirements established at the beginning, following the minimalistic
approach, providing a fully functional static site generator.
Hence, this project serves as a proof that selecting the correct tools, frameworks, language, libraries and modules,
by setting a realistic set of requirements, with the right approach that ties up the development process without
limiting its creativity, and understanding the balance between proposed design and implementation changes, any software
project can be developed with ease, ending up with a robust, flexible and open to extension results.

From the point of view of users, VaGo is ready to be used for those looking for a simple software that can allow them
to focus on content creation rather than software development details.
Create markdown content, input a couple of commands and have your static file served site ready to go.


\end{document}
