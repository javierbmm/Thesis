\markdownRendererDocumentBegin
\markdownRendererHeadingThree{Eleventy}\markdownRendererInterblockSeparator
{}Eleventy is a highly adaptable website site generator that enables you to create a website using a variety of template languages with a multitude of settings and pre-installed features to enhance and personalize the working process and final results.\markdownRendererInterblockSeparator
{}It works with JavaScript, specifically NodeJS, making its installation a primary requirement. Once it's available in the environment, use \markdownRendererCodeSpan{npm} to install it by doing \markdownRendererCodeSpan{npm install @11ty/eleventy}.\markdownRendererInterblockSeparator
{}In order to start working on the new project, it's required to create a template file, which can be accomplish by using HTML, JavaScript, Markdown, Nunjucks, and more. Once a template is already established, create the content, called index file which can be done using markdown, and the content of this file will be mapped to the specified location in the given template, generating the output files following this structure.\markdownRendererInterblockSeparator
{}Eleventy is easy-to-use, framework agnostic as it does not rely on a specific framework, unlike Next.JS and VuePress, and comes with a plethora of features, template languages, flexibility, and the ability to add multiple well-known plugins and third-party tools, such as Sass for styling, to enhance the development experience. This flexibility enables the developer to begin working on the site with the specific set of desired tools, without additional boilerplate or the burden of learning new technologies, while also adapting itself to the needs of the site to be generated, i.e. bringing the ability to use and implement already known tooling to address very specific situations in accordance with the industry standard.\markdownRendererInterblockSeparator
{}Nonetheless, it is well-known that this flexibility increases the cognitive load associated with starting a new project, facing a blank sheet of paper (or in this case, a blank IDE), and deciding which tools to use to optimize both the work process and the final product/output. The absence of an established structure can be problematic when deciding which features to include, and it makes it more difficult to work on a project that has already begun because the tools used may be entirely different from those used on a previous project.\markdownRendererDocumentEnd