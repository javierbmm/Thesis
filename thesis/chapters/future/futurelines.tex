%! Author = javif
%! Date = 9/12/2023

\chapter{Future lines}\label{ch:futurelines}

Although VaGo is capable of performing the task evaluated in this project, it is far from perfect, as there are some
improvements and features that can be added as a static site generator.

For instance, the software requires the whole code to be in the same parent folder as the input and output content,
styles, template, configuration and more.
Thus, it would be very handy to have a feature that allows the creation of a new VaGo project folder from scratch,
only with the required folders and files needed to start crafting content, without the code being in the same
directory, as it would allow the creation of multiple projects with the same code folder in another place in the
system, allowing the user to have a separation of concerns between the content and VaGo code base, useful for further
arrangement and manipulation.

On the other hand, given that it uses the concept of themes for styling, a theme package manager can be created along
with the main project to provide extended capabilities for theme publication and sharing, as well as download and
install other themes.
This will allow the user to follow the philosophy of only focusing on content creation and theme customization, as it
would subtract the need of crafting CSS styles files from scratch.

Finally, in order to meet the current state of the art of the most popular static site generators, VaGo must have the
capability of extending its functionalities via plugins, which must be easy to find, install, use and manage.
An example of a handy plugin, is the capability of adding LaTeX-like content writing, allowing the user to not only
create files on markdown format, but also extending these to use LaTex nomenclature, which is useful for
mathematicians as they can easily compose complex formulas.
This would require a plugin manager, to navigate through existing plugins and install them on command.