%! Author = javif
%! Date = 9/13/2023


\chapter{Introduction}\label{ch:introduction}

Static site generators (SSGs)\cite{wikissg,cloudflare} have emerged as a crucial component of modern web development,
providing
numerous
benefits including speed, security, and ease of website administration. In recent years, their role in simplifying
the conversion of dynamic web content to static web pages has garnered widespread recognition. This thesis explores
the development of an SSG from inception to implementation, guided by a structured software development lifecycle
(SDLC), to contribute to the ever-changing landscape of SSGs.

This research is motivated by the constant evolution of web technologies and the rising demand for effective,
customizable, and user-friendly web development tools. This thesis aims to address the limitations and
inefficiencies
observed in current SSGs by systematically advancing through the stages of the software development life cycle (SDLC)
, including a review of the state of the art, defining rigorous requirements, proposing a robust design, implementing
the software, and evaluating the results.

This study commences with a comprehensive analysis of the current state of the art in static site generators. It will
evaluate the strengths, weaknesses, and compatibility of existing SSGs with contemporary web development
requirements\cite{khalid}
. This analysis not only provides a foundation for comprehending the extant landscape, but also identifies gaps and
improvement opportunities.

Following the evaluation, the research continues with the formulation of SRS (SRS Specifications). The SRS will
establish a precise list of requirements that this SSG must satisfy. These requirements will cover a broad
range of factors, including performance, security, extensibility, and user-friendliness, and will provide a detailed
road map for the development process.

The subsequent phase entails proposing a design that incorporates innovative solutions to resolve the identified
limitations and meet the specified requirements. During the design phase, architectural patterns, data models, and
the incorporation of modern web technologies will be considered in order to expand the SSG's capabilities.

After establishing a well-defined design, the thesis will proceed to the implementation phase. This phase will
involve the actual coding and development of the SSG, adhering to industry standards and software engineering
principles. Detailed documentation will accompany the implementation to assure transparency and reproducibility.

Finally, the thesis will conclude with a thorough examination of the results obtained from the enhanced SSG's
development. These results will be analyzed critically to determine how well the objectives and requirements were met
. Throughout the development lifecycle, any deviations, obstacles, and lessons learned will be highlighted, providing
valuable insights for future SSG enhancements and web development projects.

In conclusion, the objective of this thesis is to contribute to the field of web development by meticulously
enhancing the capabilities of static site generators. This research seeks to create a sophisticated and adaptable SSG
that aligns with the evolving needs of web developers, thereby enhancing the realm of static web content creation and
management.


