\markdownRendererDocumentBegin
\markdownRendererHeadingTwo{Overall description}\markdownRendererInterblockSeparator
{}\markdownRendererHeadingThree{Product perspective}\markdownRendererInterblockSeparator
{}\markdownRendererCodeSpan{VaGo} is not the first SSG, nor does it claim to be unique or break any industry standards. Instead, it focuses on providing a simple interface for both the content creator (intended to work primarily in markdown files) and the theme author to create and style websites (meant to provide default styles while deciding editable parameters).\markdownRendererInterblockSeparator
{}In this regard, it is very similar to, and in fact inspired by, HuGo, another SSG written in the Go programming language. Despite this, \markdownRendererCodeSpan{VaGo} attempts to adopt a comparable theming methodology while incorporating simplified modification capabilities. This would enable prospective users to reuse and personalize pre-existing themes without the need to delve into CSS files. Furthermore, the system in question lacks a focus on performance and does not aim to rival HuGo in this regard. HuGo is renowned for its exceptional performance in generating and delivering static content.\markdownRendererInterblockSeparator
{}In contrast to NextJS and VuePress, which utilize React and Vue, respectively, for component creation and usage, \markdownRendererCodeSpan{VaGo} does not employ a framework-based component system. In contrast, \markdownRendererCodeSpan{VaGo} restricts itself to utilizing only native HTML, JS, and CSS. However, this does not necessarily preclude the platform from leveraging Go to facilitate or circumvent boilerplate when converting content into the aforementioned web technologies.\markdownRendererInterblockSeparator
{}Additionally, the product is not intended to incorporate any reactive techniques or external frontend frameworks in order to achieve similar results. The entirety of the content is intended to be loaded and produced on the server side, with no additional features on the client side, unless the user chooses to incorporate them using JavaScript. It is noteworthy that the utilization of opinionated simplicity fulfills the objective of facilitating content creation for users, as they are not required to attend to such particulars. However, this approach is accompanied by the disadvantage of restricted customization options.\markdownRendererInterblockSeparator
{}Furthermore, it should be noted that initially, there will be no available modules or tools for integration with this particular static site generator (SSG) framework, unlike other frameworks such as Astro. This is because the development of such modules is not a primary focus of the framework's feature development. Additionally, there are no plans to introduce a module customization feature, as outlined in the accompanying documentation.\markdownRendererInterblockSeparator
{}Ultimately, it is imperative that end-users possess the capability to engage with the product through a straightforward Command Line Interface (CLI) to produce static content and subsequently distribute it online, with minimal parameters and concealed features. Ideally, the system should be designed to facilitate user comprehension and operation without necessitating the perusal of a cumbersome 30-page manual replete with convoluted directives and extraneous verbiage. Thus, it is apparent that a minimal number of commands, typically two or three, accompanied by fundamental parameters such as port, testing mode, and designated source folder, should suffice to initiate the task at hand.\markdownRendererDocumentEnd