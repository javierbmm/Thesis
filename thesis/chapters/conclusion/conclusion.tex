%! Author = javif
%! Date = 9/12/2023


\chapter{Conclusion}\label{ch:conclusion}

As demonstrated on previous chapter, VaGo is a software capable of obtaining input markdown files from an established
folder, read each file content, navigate through the abstract syntax tree, obtain the required tokens to build up a
web page, parse it into respective HTML tags, create HTML files accordingly to the input content using the same name,
build up dedicated styles using a theme system reading customization variables, read configuration file to adapt
system to user specifications, provide a listener server on a provided port, route requests to map the same file name
convention to target URLs, display insightful logging and an intuitive command interface with discerning usage
documentation.


As a result, it has been shown that the implementation of VaGo successfully meets the requirements outlined in
previous sections.
Furthermore, it can be regarded as a favorable alternative to existing Static Site Generators, as
it possesses the ability to execute a majority of the characteristics commonly observed in the present market.
This is without accounting for some other measures that are not the primary emphasis of this project, such as its
lightning-quick creation and serving of files, which opens up new possibilities beyond merely serving static
information: The rapid pace of building construction facilitates the expansion of serverside rendering capabilities.


What's more, VaGo has been developed using a minimalistic approach, reducing the code and the cognitive load to the
lowest possible, with a balance between fulfilling requirements and following the proposed design with the desired
simplicity.
This minimalistic philosophy has demonstrated an overall increased productivity in terms of implementation, while
reducing the amount of complications and challenges to face when dealing with bugs and errors, with the addition of
being an open door to extension given that the code base is very simple and easy to read and understand, allowing
future collaborators to improve its functioning and/or add new features.

All in all, VaGo has been capable of meeting the requirements established at the beginning, following the minimalistic
approach, providing a fully functional static site generator.
Hence, this project serves as a proof that selecting the correct tools, frameworks, language, libraries and modules,
by setting a realistic set of requirements, with the right approach that ties up the development process without
limiting its creativity, and understanding the balance between proposed design and implementation changes, any software
project can be developed with ease, ending up with a robust, flexible and open to extension results.

From the point of view of users, VaGo is ready to be used for those looking for a simple software that can allow them
to focus on content creation rather than software development details.
Create markdown content, input a couple of commands and have your static file served site ready to go.
