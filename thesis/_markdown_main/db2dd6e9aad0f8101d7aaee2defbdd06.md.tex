\markdownRendererDocumentBegin
\markdownRendererHeadingTwo{Overall description}\markdownRendererInterblockSeparator
{}\markdownRendererHeadingThree{Product perspective}\markdownRendererInterblockSeparator
{}\markdownRendererCodeSpan{VaGo} is not the first SSG, nor does it claim to be unique or break any industry standards. Instead, it focuses on providing a simple interface for both the content creator (intended to work primarily in markdown files) and the theme author to create and style websites (meant to provide default styles while deciding editable parameters).\markdownRendererInterblockSeparator
{}In this regard, it is very similar to, and in fact inspired by, HuGo, another SSG written in the Go programming language. Despite this, \markdownRendererCodeSpan{VaGo} attempts to adopt a comparable theming methodology while incorporating simplified modification capabilities. This would enable prospective users to reuse and personalize pre-existing themes without the need to delve into CSS files. Furthermore, the system in question lacks a focus on performance and does not aim to rival HuGo in this regard. HuGo is renowned for its exceptional performance in generating and delivering static content.\markdownRendererInterblockSeparator
{}In contrast to NextJS and VuePress, which utilize React and Vue, respectively, for component creation and usage, \markdownRendererCodeSpan{VaGo} does not employ a framework-based component system. In contrast, \markdownRendererCodeSpan{VaGo} restricts itself to utilizing only native HTML, JS, and CSS. However, this does not necessarily preclude the platform from leveraging Go to facilitate or circumvent boilerplate when converting content into the aforementioned web technologies.\markdownRendererInterblockSeparator
{}Additionally, the product is not intended to incorporate any reactive techniques or external frontend frameworks in order to achieve similar results. The entirety of the content is intended to be loaded and produced on the server side, with no additional features on the client side, unless the user chooses to incorporate them using JavaScript. It is noteworthy that the utilization of opinionated simplicity fulfills the objective of facilitating content creation for users, as they are not required to attend to such particulars. However, this approach is accompanied by the disadvantage of restricted customization options.\markdownRendererInterblockSeparator
{}Furthermore, it should be noted that initially, there will be no available modules or tools for integration with this particular static site generator (SSG) framework, unlike other frameworks such as Astro. This is because the development of such modules is not a primary focus of the framework's feature development. Additionally, there are no plans to introduce a module customization feature, as outlined in the accompanying documentation.\markdownRendererInterblockSeparator
{}Ultimately, it is imperative that end-users possess the capability to engage with the product through a straightforward Command Line Interface (CLI) to produce static content and subsequently distribute it online, with minimal parameters and concealed features. Ideally, the system should be designed to facilitate user comprehension and operation without necessitating the perusal of a cumbersome 30-page manual replete with convoluted directives and extraneous verbiage. Thus, it is apparent that a minimal number of commands, typically two or three, accompanied by fundamental parameters such as port, testing mode, and designated source folder, should suffice to initiate the task at hand.\markdownRendererInterblockSeparator
{}\markdownRendererHeadingThree{Product functions}\markdownRendererInterblockSeparator
{}This Static Site Generator's primary objective is to convert markdown files' text to plain HTML, mapping the markup elements to understandable web elements in order to maintain the Web Standards' accessibility.\markdownRendererInterblockSeparator
{}However, it should also be able to offer a theme system built on CSS files that gives the HTML elements styling. This will be made possible by also producing CSS with a set of additional parameters that follow a pre-established structure and should be simple to alter using Go variables.\markdownRendererInterblockSeparator
{}Last but not least, \markdownRendererCodeSpan{VaGo} will have a brief set of parameters to enable communication with the user via CLI, to enable them to create static content (translation of markdown to HTML with the addition of a theme via CSS), and to serve content via a particular port. A set of features for exporting and importing themes from authors whose works are shared online and/or in repositories may be added in the future.\markdownRendererInterblockSeparator
{}\markdownRendererHeadingThree{User characteristics}\markdownRendererInterblockSeparator
{}\markdownRendererCodeSpan{VaGo} is designed to be utilized by individuals possessing a basic understanding of web development and terminal operations. However, this does not necessarily demand that they must hold a degree in Computer Science in order to effectively employ the system. The primary objective of the system is to facilitate ease of use for individuals with limited knowledge of computer technologies, including academic scientists seeking to disseminate their knowledge, blog posts, papers, and ideas on the internet, without requiring extensive knowledge of web technologies.\markdownRendererInterblockSeparator
{}Therefore, a computer-based academic background is not needed for this system to be used, as long as there is a bit of knowledge about markdown and how to use a terminal, it should be entirely enough. Henceforth, individuals possessing a greater depth of knowledge would be capable of executing more intricate undertakings by leveraging the available adaptability while adhering to the prescribed limitations.\markdownRendererInterblockSeparator
{}\markdownRendererHeadingThree{Constraints}\markdownRendererInterblockSeparator
{}Considering that \markdownRendererCodeSpan{VaGo} is programmed using the Go language, which inherently supports cross-platform functionality, the development efforts will be concentrated solely on Linux to streamline the implementation of CLI features and circumvent potential complications arising from newline interpretation and other cross-platform limitations.\markdownRendererInterblockSeparator
{}In addition, the system does not account for security considerations related to delivering content over the internet, and therefore it cannot guarantee any level of safety in this regard beyond what is already provided by the Go programming language.\markdownRendererInterblockSeparator
{}However, while there exist fundamental performance considerations for its implementation, it neither guarantees nor rivals other frameworks with respect to performance, velocity, or optimal resource utilization.\markdownRendererInterblockSeparator
{}Consequently, the implementation and development of \markdownRendererCodeSpan{VaGo} will primarily rely on the pre-existing features of the language. As such, it is not anticipated that \markdownRendererCodeSpan{VaGo} will establish its own mechanisms for managing HTTP requests, signaling protocols for handshaking, thread management through Go routines, or other hardware-related dependencies and controls for the host computer. Therefore, any inherent limitations within the Go programming language will inevitably impact the system, and no resources will be allocated towards attempting to circumvent them.\markdownRendererDocumentEnd