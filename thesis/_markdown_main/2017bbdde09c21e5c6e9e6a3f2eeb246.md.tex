\markdownRendererDocumentBegin
\markdownRendererHeadingTwo{Introduction}\markdownRendererInterblockSeparator
{}\markdownRendererHeadingThree{Purpose}\markdownRendererInterblockSeparator
{}The primary objective of this chapter, which is referred to as the Software Requirements Specification (from this point forward, SRS), is to define the requirements and, as a result, the goals that need to be accomplished by the Static Site Generator that is described and implemented throughout the entirety of this project. This will be accomplished by providing a detailed explanation of how the system as a whole works as well as the various components that make up the system.\markdownRendererInterblockSeparator
{}The person in charge of the creation of the SSG, the developer and/or implementer, is the intended audience for this SRS. Given that it outlines the considered implementation details and design specifications, it serves as a guide for the developer to know exactly what to do along with the definition of done for each task and/or component.\markdownRendererInterblockSeparator
{}\markdownRendererHeadingThree{Scope}\markdownRendererInterblockSeparator
{}The Static Site Generator described in the document will be known as \markdownRendererCodeSpan{VaGo}, and will henceforth be referred to as such. The nomenclature of this tool is derived from its implementation in the Go programming language, as expounded upon in subsequent sections. Additionally, the name incorporates a homograph word in Spanish, "vago," which connotes indolence or sloth in English. This serves as a playful allusion to the minimal exertion demanded of prospective users in generating a static website through utilization of this software.\markdownRendererInterblockSeparator
{}VaGo is a tool designed to facilitate the creation and distribution of static content on the web. Its primary function is to interpret and translate markdown files into web content, which includes HTML, CSS, and necessary JavaScript. Furthermore, the system will facilitate the implementation of a theming mechanism, which will enable the creation and dissemination of particular style guidelines for reuse and adherence by the content after its translation from markdown. This theming system will also allow for a flexible and open approach to effectuate specific author-provided modifications to the style of the pages, such as the inclusion of particular padding between elements, the adjustment of text size based on the heading hierarchy, the selection of colors, the application of specific button styles, and other customizable features, which will be contingent upon the preferences established by the theme author.\markdownRendererInterblockSeparator
{}The provision of flexibility enables prospective users of designated themes to make necessary adjustments to suit their individual requirements, without the need of extensively search through the styling boilerplate and navigating through numerous CSS files to locate the particular parameter to modify.\markdownRendererInterblockSeparator
{}Similarly, VaGo will offer a straightforward command-line interface (CLI) to facilitate interaction with the program to serve the purpose of translating content into static web pages. This will be possible once the content has been provided in the form of markdown, along with theme styling. Additionally, VaGo will enable users (via the command line) to serve the content as websites through a specified port, using Go's native HTTP methods.\markdownRendererInterblockSeparator
{}Conversely, the system won't offer support for alternative interpretation formats, such as YAML or JSON, nor can be expected it facilitates the translation of Go code to JavaScript or vice versa. Hence, it is not to be anticipated that it would provide backing for JavaScript reactivity, akin to what is observed in other front-end frameworks. VaGo will exclusively restrict its capabilities to the delivery of static content through fundamental and native CSS and JavaScript. Its primary objective is to facilitate the development process, with a singular focus on this goal to ensure optimal performance. This approach aims to alleviate the burden of web development for individuals who seek to share text or images without delving into the intricacies of the field.\markdownRendererInterblockSeparator
{}\markdownRendererHeadingThree{Definitions, acronyms and abbreviations.}\markdownRendererInterblockSeparator
{}\markdownRendererUlBeginTight
\markdownRendererUlItem SSG: Static Site Generator.\markdownRendererUlItemEnd 
\markdownRendererUlItem MD: Markdown files.\markdownRendererUlItemEnd 
\markdownRendererUlItem JS: Javascript.\markdownRendererUlItemEnd 
\markdownRendererUlItem CSS: Cascading Style Sheets.\markdownRendererUlItemEnd 
\markdownRendererUlItem HTML: HyperText Markup Language.\markdownRendererUlItemEnd 
\markdownRendererUlItem CLI: Command-line interface.\markdownRendererUlItemEnd 
\markdownRendererUlItem Theme/Style theme(ing): A set of established reusable guidelines and rules to provide style to the web content (website) via HTML and CSS.\markdownRendererUlItemEnd 
\markdownRendererUlEndTight \markdownRendererInterblockSeparator
{}\markdownRendererHeadingThree{References}\markdownRendererInterblockSeparator
{}TODO: This section is empty hence must be defined or completed once reference are found. \markdownRendererInterblockSeparator
{}\markdownRendererHeadingThree{Overall description}\markdownRendererInterblockSeparator
{}TODO: Add a short description of the whole structure of the document (SRS).\markdownRendererDocumentEnd