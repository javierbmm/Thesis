\markdownRendererDocumentBegin
\markdownRendererHeadingThree{NextJS}\markdownRendererInterblockSeparator
{}NextJS is a NodeJS with ReactJS based SSG framework, meaning that in order to use it's needed to install the following dependencies.\markdownRendererInterblockSeparator
{}\markdownRendererOlBeginTight
\markdownRendererOlItemWithNumber{1}First, install NodeJS: \markdownRendererCodeSpan{sudo apt install nodejs}\markdownRendererOlItemEnd 
\markdownRendererOlItemWithNumber{2}Then, install npm (Node package manager): \markdownRendererCodeSpan{sudo apt install npm}\markdownRendererOlItemEnd 
\markdownRendererOlItemWithNumber{3}Finally, using npm to install NextJS: \markdownRendererCodeSpan{npx create-next-app@latest nextjs-blog —use-npm}\markdownRendererOlItemEnd 
\markdownRendererOlEndTight \markdownRendererInterblockSeparator
{}This command will install the additional dependencies required to develop a NextJS application, such as React. The pages of this SSG are located in the /pages folder, and each page entrypoint has the same name as its corresponding file. As a result, routing is handled automatically and everything is well-defined within the project.\markdownRendererInterblockSeparator
{}There is no need to switch to a different file to view the routing pages or the names, as one folder contains all of this information.\markdownRendererInterblockSeparator
{}In addition, the fundamental feature of NextJS is that it uses React as a foundation framework to generate content via components, and the page itself is a React component, so it's really simple to comprehend and begin working with if you' re already familiar with React.\markdownRendererInterblockSeparator
{}The styles, on the other hand, are covered by basic CSS files in a styles/ folder, as is customary in web development.\markdownRendererInterblockSeparator
{}Overall, I found NextJS to be incredibly user-friendly, simple to design, utilize, and scale, and well-structured.\markdownRendererDocumentEnd