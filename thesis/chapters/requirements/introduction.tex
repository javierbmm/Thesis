%! Author = javif
%! Date = 9/14/2023

\section{Introduction}\label{subsec:ssg-introduction}

\subsection{Purpose}\label{subsubsec:purpose}

The primary objective of this chapter, which is referred to as the
Software Requirements Specification\cite{srs} (from this point forward, SRS), is
to define the requirements and, as a result, the goals that need to be
accomplished by the Static Site Generator\cite{cloudflare,wikissg} that is described and
implemented throughout the entirety of this project. This will be
accomplished by providing a detailed explanation of how the system as a
whole works as well as the various components that make up the system.

The person in charge of the creation of the SSG, the developer and/or
implementer, is the intended audience for this SRS. Given that it
outlines the considered implementation details and design
specifications, it serves as a guide for the developer to know exactly
what to do along with the definition of done for each task and/or
component.

\subsection{Scope}\label{subsubsec:scope}

The Static Site Generator described in the document will be known as
\texttt{VaGo}, and will henceforth be referred to as such. The
nomenclature of this tool is derived from its implementation in the Go
programming language\cite{donovan2015go}, as expounded upon in subsequent sections.
Additionally, the name incorporates a homograph word in Spanish,
``vago'', which connotes indolence or sloth in English. This serves as a
playful allusion to the minimal exertion demanded of prospective users
in generating a static website through utilization of this software.

VaGo is a tool designed to facilitate the creation and distribution of
static content on the web. Its primary function is to interpret and
translate markdown files into web content, which includes HTML, CSS, and
necessary JavaScript. Furthermore, the system will facilitate the
implementation of a theming mechanism, which will enable the creation
and dissemination of particular style guidelines for reuse and adherence
by the content after its translation from markdown. This theming system
will also allow for a flexible and open approach to effectuate specific
author-provided modifications to the style of the pages, such as the
inclusion of particular padding between elements, the adjustment of text
size based on the heading hierarchy, the selection of colors, the
application of specific button styles, and other customizable features,
which will be contingent upon the preferences established by the theme
author.

The provision of flexibility enables prospective users of designated
themes to make necessary adjustments to suit their individual
requirements, without the need of extensively search through the styling
boilerplate and navigating through numerous CSS files to locate the
particular parameter to modify.

Similarly, VaGo will offer a straightforward command-line interface
(CLI)\cite{wikicli} to facilitate interaction with the program to serve the purpose of
translating content into static web pages. This will be possible once
the content has been provided in the form of markdown, along with theme
styling. Additionally, VaGo will enable users (via the command line) to
serve the content as websites through a specified port, using Go's
native HTTP methods.

Conversely, the system won't offer support for alternative
interpretation formats, such as YAML or JSON, nor can be expected it
facilitates the translation of Go code to JavaScript or vice versa.
Hence, it is not to be anticipated that it would provide backing for
JavaScript reactivity, akin to what is observed in other front-end
frameworks. VaGo will exclusively restrict its capabilities to the
delivery of static content through fundamental and native CSS and
JavaScript. Its primary objective is to facilitate the development
process, with a singular focus on this goal to ensure optimal
performance. This approach aims to alleviate the burden of web
development for individuals who seek to share text or images without
delving into the intricacies of the field.

\subsection{Definitions, acronyms and
abbreviations.}\label{subsubsec:definitions-acronyms-and-abbreviations.}

\begin{itemize}
    \item
    SSG: Static Site Generator.
    \item
    MD: Markdown files.
    \item
    JS: Javascript.
    \item
    CSS: Cascading Style Sheets.
    \item
    HTML: HyperText Markup Language.
    \item
    CLI: Command-line interface.
    \item
    Theme/Style theme(ing): A set of established reusable guidelines and
    rules to provide style to the web content (website) via HTML and CSS
    for the sake of consistency.
\end{itemize}

\subsection{References}\label{subsubsec:srs-references}

The following collection provides a selection of research documents
aimed at facilitating comprehension of the foundational principles that
motivate the prerequisites for VaGo. Emphasis is placed on elucidating
the rationale behind the necessity of these requirements.


\begin{itemize}
    \item
    \emph{What is a static site generator?}. Cloudflare. (n.d.) .
    https://www.cloudflare.com/learning/performance/static-site-generator/
    \item
    Khalid, F. S. (2022, April 18). \emph{How to choose the right static
    site generator}. GitLab.
    https://about.gitlab.com/blog/2022/04/18/comparing-static-site-generators/
    \item
    Wikimedia Foundation. (2023, August 12). \emph{Static Site Generator}.
    Wikipedia. https://en.wikipedia.org/wiki/Static\_site\_generator
    \item
    Wikimedia Foundation. (2023a, April 11). \emph{Hugo (software)}.
    Wikipedia. https://en.wikipedia.org/wiki/Hugo\_(software)
    \item
    \emph{What is Hugo}. Hugo. (2023, July 13).
    https://gohugo.io/about/what-is-hugo/
    \item
    \emph{Docs. What is Next.js}. Next.js. (n.d.). https://nextjs.org/docs
    \item
    \emph{VuePress guide. Introduction}. VuePress. (n.d.).
    https://v2.vuepress.vuejs.org/guide/
    \item
    \emph{Accessibility principles}. Web Accessibility Initiative (WAI) .
    https://www.w3.org/WAI/fundamentals/accessibility-principles/
    \item
    MozDevNet. (n.d.). \emph{What is accessibility?} Learn web development
    \textbar{} MDN.
    https://developer.mozilla.org/en-US/docs/Learn/Accessibility/What\_is\_accessibility
    \item
    MozDevNet. (n.d.-a). \emph{HTML: A good basis for accessibility.}
    Learn web development \textbar{} MDN.
    https://developer.mozilla.org/en-US/docs/Learn/Accessibility/HTML
    \item
    Sun, Y. (2019). \emph{Server-Side Rendering. In: Practical Application
    Development with AppRun}. Apress, Berkeley, CA.
    https://doi.org/10.1007/978-1-4842-4069-4\_9
    \item
    Taufan Fadhilah Iskandar et al (2020). \emph{Comparison between
    client-side and server-side rendering in the web development}
    https://iopscience.iop.org/article/10.1088/1757-899X/801/1/012136/pdf
\end{itemize}

