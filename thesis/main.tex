%! Author = Javier Mérida
%! Date = 2/12/2023

% Preamble
\documentclass[12pt]{report}
\parskip=\baselineskip
\setcounter{secnumdepth}{4}
% Packages
\usepackage[utf8]{inputenc}
\usepackage{graphicx}
\usepackage[smartEllipses]{markdown}
\usepackage{setspace}
\usepackage[table,xcdraw]{xcolor}
\usepackage{float}
\usepackage{url}
\usepackage[nottoc,notlot,notlof]{tocbibind}
\singlespacing
% Graphics
\graphicspath{ {images/} }
\floatplacement{figure}{H}
% Verbatim
\usepackage{framed,color,verbatim}
\usepackage{textcomp}
\usepackage[hidelinks]{hyperref}
\definecolor{shadecolor}{rgb}{.9, .9, .9}
\newenvironment{code}%
{\snugshade\verbatim}%
{\endverbatim\endsnugshade}
% title
\title{
    \vspace*{1cm}
    {\includegraphics{lasalle-logo}}
    {Static Site Generator}\\
    {\large La Salle Campus Barcelona}\\
}
\author{Javier Mérida}
\date{12 02 2023}


% Document
\begin{document}

%%    \maketitle
%    \begin{titlepage}
%        \begin{center}
%%            \vspace*{1cm}
%            \includegraphics[width=0.7\textwidth]{lasalle-logo}
%
%        \end{center}
%        \textbf{Static Site Generator}
%
%        \vspace{0.5cm}
%        An study of software design implementation following a minimalistic approach.
%
%        \vspace{1.5cm}
%
%        \textbf{Javier Baltazar Merida Morales}
%
%        \vfill
%
%        A thesis presented for the degree of\\
%        International Computer Engineering
%
%        \vspace{0.8cm}
%
%
%
%        Department Name\\
%        University Name\\
%        Country\\
%        Date
%
%    \end{titlepage}

    \chapter*{Abstract}\label{ch:ch:chapter}
    \section*{English}\label{sec:english}
    This project presents the development of a Static Site Generator. It will start with explanations and insights
    into what a Static Site Generator (SSG) is, and what is expected from it, with a study of the current state of
    the art, then evaluating what features will be considered for the development of a new one, based on what is
    currently available, and from this point a set of requirements will be built, using the Software Requirement
    Specification IEEE document guide, to establish what could be expected and/or required from it. After this, the
    document will explore a software design proposal, which may or not be followed in the implementation as further
    changes may be needed during development. Finally, it will reach some conclusions based on what has been achieved
    and whether the proposed approach was correct, or if there is any improvement that could be made or evaluated.

    \section*{Español}
    Este proyecto presenta el desarrollo de un Generador de Sitios Estáticos. Comenzará con explicaciones y
    perspectivas sobre qué es un Generador de Sitios Estático (SSG), qué se espera de él, con un estudio del estado
    actual del arte, para luego evaluar qué características se considerarán para el desarrollo de uno nuevo, basado
    en lo que está disponible actualmente, y a partir de este punto se construirá un conjunto de requisitos,
    utilizando la guía del documento de Especificación de requisitos de software IEEE, para establecer lo que se podría
    esperar y/o requerir del mismo. Después de esto, el documento explorará una propuesta de diseño de software, que
    puede seguirse o no en la implementación, ya que es posible que se necesiten más cambios durante el desarrollo.
    Finalmente, se llegará a algunas conclusiones en base a lo logrado, y si el enfoque propuesto fue correcto, o si
    existe alguna mejora que se pueda realizar o evaluar.

    \section*{Català}
    Aquest projecte presenta el desenvolupament d'un generador de llocs estàtics. Començarà amb explicacions i
    coneixements sobre què és un generador de llocs estàtics (SSG) i què se n'espera, amb un estudi de l'estat actual
    de la tècnica, i després avaluant quines característiques es tindran en compte per al desenvolupament d'un de nou
    . , basant-se en el que està disponible actualment, i a partir d'aquest punt es construirà un conjunt de
    requisits, utilitzant la guia de documents IEEE Software Requirement Specification, per establir què es podria
    esperar i/o requerir d'aquest nou model. Després d'això, el document explorarà una proposta de disseny de
    programari, que es pot seguir o no en la implementació, ja que poden ser necessaris més canvis durant el
    desenvolupament. Finalment, s'arriba a unes conclusions en funció del que s'ha aconseguit i de si l'enfocament
    proposat era correcte, o si hi ha alguna millora que es podria fer o avaluar.

    \tableofcontents

    %! Author = javif
%! Date = 9/13/2023


\chapter{Introduction}\label{ch:introduction}

This project will explore the design solution proposal and implementation for an
SSG~\cite{cloudflare}~\cite{hugo}~\cite{vuepress}~\cite{accwai}



    \chapter{Trying out popular SSGs}\label{ch:trying-out-popular-ssgs}

    The following SSGs has been chosen based on their popularity in order
    to be tried to gather information on what the basic features for SSG
    are.
    It's important to notice that these are being tried in a Linux environment
    (Linux inside Windows, thanks to WSL), installing them from scratch, using vim
    as text editor to minimize external tools assistance to have an unbiased appreciation
    of each one of them.

    \begin{itemize}
        \item NextJS (https://nextjs.org/)
        \item Hugo\cite{hugo}
        \item VuePress
        \item Eleventy
        \item Astro
    \end{itemize}

    % Trying NextJS
    \markdownInput{chapters/testing-ssgs/nextjs.md}

    % Trying Hugo
    \markdownInput{chapters/testing-ssgs/hugo.md}

    % Trying VuePress
    \markdownInput{chapters/testing-ssgs/vuepress.md}

    % Trying Eleventy
    \markdownInput{chapters/testing-ssgs/eleventy.md}

    % Trying Astro
    \markdownInput{chapters/testing-ssgs/astro.md}


    \section{Comparison Table}\label{sec:comparison-table}


    The following table outlines the features that has been considered as most important to take into account when
    considering static site generator.
    Note that the purpose of this document is not to benchmark any of these features,
    thus there is not a workload test to compare the performance of the explored SSGs. Instead, this is a simple
    exploration on what features are stand out in order to consider as requisites to build a new SSG.


    Also, it is important to note that the lack or the presence of any of these characteristic do not make any SSG
    better
    than others, as each one of them are made for a specific purpose and shine in terms of addressing the obstacle
    they are meant to\cite{khalid}.
    In fact, this section must be taken as an exploration of requisites to be taken into account.


    % Please add the following required packages to your document preamble:
% \usepackage[table,xcdraw]{xcolor}
% If you use beamer only pass "xcolor=table" option, i.e. \documentclass[xcolor=table]{beamer}
    \begin{table}[H]
        \begin{tabular}{|
                >{\columncolor[HTML]{EFEFEF}}c |c|c|c|c|c|}
            \hline
            \cellcolor[HTML]{C0C0C0} & \cellcolor[HTML]{EFEFEF}NextJS & \cellcolor[HTML]{EFEFEF}Hugo &
            \cellcolor[HTML]{EFEFEF}VuePress & \cellcolor[HTML]{EFEFEF}Eleventy & \cellcolor[HTML]{EFEFEF}Astro \\
            \hline
            \textbf{Easy to use} & Yes & Yes & No
            & Yes & Yes \\ \hline
            \textbf{Flexible} & No & No & No
            & Yes & Yes \\ \hline
            \textbf{Strict structure} & No & Yes & No
            & No & No \\ \hline
            \textbf{Template system} & No & Yes & No
            & Yes & Yes \\ \hline
            \textbf{Themes} & No & Yes & No
            & Yes & Yes \\ \hline
            \textbf{Performance oriented} & Yes & Yes & NA
            & Yes & Yes \\ \hline
            \textbf{Framework dependant} & Yes & No & Yes
            & No & No \\ \hline
            \textbf{Easy to install} & Yes* & Yes & Yes*
            & Yes* & Yes* \\ \hline
            \textbf{File-based routing} & Yes & Yes & NA
            & Yes & Yes \\ \hline
        \end{tabular}\label{tab:table}
    \end{table}


    Note*: The installation difficulty is strickly dependant of NodeJS and how `npm` addresses any missing dependency,
    which can be (sometimes) a painful issue to solve.

    % SRS


    \chapter{Software requirements specifications}\label{ch:software-requirements-specifications}

    % SRS Introduction.
    %\markdownInput{chapters/requirements/introduction.md}
    %! Author = javif
%! Date = 9/13/2023


\chapter{Introduction}\label{ch:introduction}

This project will explore the design solution proposal and implementation for an
SSG~\cite{cloudflare}~\cite{hugo}~\cite{vuepress}~\cite{accwai}


    % SRS Overall description
    %\markdownInput{chapters/requirements/description.md}
    \input{chapters/requirements/description.tex}

    % SRS Specific requirements.
%    \markdownInput{chapters/requirements/requirements.md}
    %! Author = javif
%! Date = 9/14/2023

\section{Specific requirements}\label{sec:specific-requirements}

\subsection{External interfaces}\label{subsec:external-interfaces}

Users will be able to interact with the system using CLI commands\cite{wikicli} to
carry out simple operations that are built into the system. Here, two
fundamental commands should be noted:

\begin{itemize}
    \item
    \texttt{Build}: To parse and generate HTML content from markdown files.
    \item
    \texttt{Serve}: To serve and share content via a specific port.
\end{itemize}

Additionally, the system ought to alert the user any time a command it
has issued contains an error because it is possible for them to
accidentally type the wrong command or misspell any of these. (using
buidl instead of build). Notably, unlike other systems, it will lack a
recommendation system to offer hints of similar words that are actually
interpreted as to assist the user in finding alternative commands, even
though it will inform the user about the existence of an error that
prevented the system from performing the required task.

\subsubsection{Build}\label{subsubsec:build}

The task at hand involves the conversion of Markdown files\cite{wikimd} into
web-based content, specifically in the form of HTML\cite{htmlmoz}, JS\cite{jsmoz}, and CSS\cite{cssmoz}.

The program is designed to utilize the Command Line Interface (CLI) as
an input mechanism, whereby it recognizes the command name to execute
its designated task. Its output function involves generating the HTML
and CSS content into an \texttt{out} folder, in accordance with the
pre-existing Markdown files. It is noteworthy that each MD file ought to
generate an individual page to allow for the adaptability of
transforming every file into a distinct output for the website.

Throughout the translation process, it is imperative that the user is
kept informed of its progress. In the event of any errors, the system
should prompt error messages to ensure the accuracy of the MD files,
while simultaneously verifying that all necessary parameters for content
and styling have been provided. Upon completion, the system should
explicitly indicate the out folder path to the user.

Therefore, it should use the following format for each of the steps
considered for the translation and building process: {[}Date{]}{[}Page
Name{]}: Step, where each one of these detail token corresponds to:

\begin{itemize}
    \item
    Date: Specific date and time using the ISO 8601 format\cite{wikiiso} without specifying
    the timezone as \texttt{VaGo} will limit itself to use the same
    timezone from the host machine. For instance: \emph{2023-04-09T14:
    22:05} should be used to represent the year 2023, month April (04),
    day 09 at 14 hours, 22 minutes and 05 seconds.
    \item
    Page Name: To specify the exact page name for the URL that takes to
    that specific page, which is the same as the MD file name.
    \item
    Step: To specify the building step it's currently on, whether it is
    identifying the MD tokens\cite{schutze2008introduction, wikilexical}, performing translation to HTML and/or
    obtaining the parameters set for the theming and styling of the given
    page.
\end{itemize}

On the other hand, additional parameters can be added to the command for
specific output capabilities. For instance, \texttt{no-log} should be
using to omitting logging (no prompt message for every building process
step) except for error and finalization message. Also, \texttt{no-time}
should be used to omit the timestamp from the steps, prompting a message
with the following structure: \emph{{[}Page name{]}:Step}, with the
purpose of avoiding overloading logs and make them easier to read.

\subsubsection{Serve}\label{subsubsec:serve}

The second command serves the purpose of keep the system continuously
executing, hence no accepting other commands in the meantime, in order
to open a port and send the already built files (by means of the
previous command, \texttt{build}) via HTTP\cite{gourley2002http}.

It is important to note that this command will exclusively serve content
available in the \texttt{out} folder, therefore if the previous command
hasn't been run, or if there are no files in this folder, then it won't
be able to serve any content at all. Because of it, it is required that
the user must first build content in order to then serve it.

Moreover, in addition to the behavior noted above, it must output logs
to inform the user of the current status of the serving process,
starting by indicating as soon as it starts reading content and
initiating listening for requests, while also informing of any issue
that may occur by prompting error messages.

Once it starts listening for requests, it should inform of any incoming
request/response using the following format: {[}Date{]}: Sending
        {[}Page{]} to {[}requester IP{]}, where each one of these token
corresponds to:

\begin{itemize}
    \item
    Date: Same as the one outlined for \texttt{build} command, using the
    ISO 8601 format without timezone.
    \item
    Page: Page name as stated in the MD filename.
    \item
    Requester IP: Specific IP address of the incoming IP.
\end{itemize}

Additionally, an extra (optional) parameter can be provided to outline
the specific port the system should open to listen for requests, instead
of the default one.

\subsection{Functions}\label{subsec:functions}

The following provides a more in-depth explanation of the functional
areas, including full requirements.

\subsubsection{Generate page}\label{subsubsec:generate-page}

\paragraph{Description and priority}\label{par:description-and-priority}

The task at hand involves generating a page document through the process
of translating the contents of a Markdown file into HTML. This requires
the utilization of appropriate tags to accurately map the entities
present within each Markdown block. Priority: HIGH.

\paragraph{Stimulus/Response
sequences}\label{par:stimulusresponse-sequences}

Stimulus: This feature must obtain the markdown files from a
\texttt{source} folder.

Response: Translate or generate a corresponding HTML file for each one
of these MD files, outputting the result pages into an \texttt{out}
folder.

\paragraph{Functional requirements}\label{par:functional-requirements}

\textbf{REQ-1}: The MD blocks must be interpreted and translated to HTML
tags to generate a page accordingly to what the user has described
initially. It should, at least, implement the following elements to
ensure a fully accessible page as per best practices for web standards\cite{accmoz}:

\begin{itemize}
    \item
    Heading level 1 (\texttt{\#}): Heading 1
    (\texttt{\textless{}h1\textgreater{}\textless{}/h1\textgreater{}}).
    \item
    Heading level 2 (\texttt{\#\#}): Heading 2
    (\texttt{\textless{}h2\textgreater{}\textless{}/h2\textgreater{}}).
    \item
    Heading level 3 (\texttt{\#\#\#}): Heading 3
    (\texttt{\textless{}h3\textgreater{}\textless{}/h3\textgreater{}}).
    \item
    Heading level 4 (\texttt{\#\#\#\#}): Heading 4
    (\texttt{\textless{}h4\textgreater{}\textless{}/h4\textgreater{}}).
    \item
    Heading level 5 (\texttt{\#\#\#\#\#}): Heading 5
    (\texttt{\textless{}h5\textgreater{}\textless{}/h5\textgreater{}}).
    \item
    Heading level 6 (\texttt{\#\#\#\#\#\#}): Heading 6
    (\texttt{\textless{}h6\textgreater{}\textless{}/h6\textgreater{}}).
    \item
    Text: Paragraph
    (\texttt{\textless{}p\textgreater{}\textless{}/p\textgreater{}}).
    \item
    Bold text (\texttt{**text**}): Strong
    (\texttt{\textless{}strong\textgreater{}\textless{}/strong\textgreater{}}).
    \item
    Italic text (\texttt{\_text\_}): Emphasis
    (\texttt{\textless{}em\textgreater{}\textless{}/em\textgreater{}}).
    \item
    Blockquote (\texttt{\textgreater{}}): Blockquote
    (\texttt{\textless{}blockquote\textgreater{}\textless{}/blockquote\textgreater{}}).
    \item
    Unordered list (\texttt{-}): Unordered list
    (\texttt{\textless{}ul\textgreater{}\textless{}/ul\textgreater{}})
    surrounding list items
    (\texttt{\textless{}li\textgreater{}\textless{}/li\textgreater{}}).
    \item
    Ordered list (\texttt{1.}, \texttt{2.}, \texttt{3.}, and so): Ordered
    list
    (\texttt{\textless{}ol\textgreater{}\textless{}/ol\textgreater{}})
    surrounding list items
    (\texttt{\textless{}li\textgreater{}\textless{}/li\textgreater{}}).
    \item
    Code (\texttt{\textasciigrave{}\textasciigrave{}}): Code
    (\texttt{\textless{}code\textgreater{}\textless{}/code\textgreater{}}).
    \item
    Fenced code block
    (\texttt{\textasciigrave{}\textasciigrave{}\textasciigrave{}\ \textasciigrave{}\textasciigrave{}\textasciigrave{}}):
    Preformatted text
    (\texttt{\textless{}pre\textgreater{}\textless{}/pre\textgreater{}})
    surrounding a code block
    (\texttt{\textless{}code\textgreater{}\textless{}/code\textgreater{}}).
    \item
    Link (\texttt{{[}text{]}(URL)}): Anchor tag along with the hyperlink
    attribute
    (\texttt{\textless{}a\ href="URL"\textgreater{}text\textless{}/a\textgreater{}}).
    \item
    Image (\texttt{!{[}Alt\ text{]}(URL)}): Image tag along with alt text
    and source attributes
    (\texttt{\textless{}img\ alt="Alt\ text"\ src="URL"\textgreater{}\textless{}/img\textgreater{}}).
\end{itemize}

\textbf{REQ-2}: It must verify whenever there is an input error or
incorrect usage of the MD blocks, and thereby let the user know about
these via error messages and halt the translation process immediately.

\textbf{REQ-3}: Each generated page will have the same file name as the
input MD file.

\subsubsection{Theme styling creation and
customization}\label{subsubsec:theme-styling-creation-and-customization}

\paragraph{Description and
priority}\label{par:description-and-priority-1}

This feature is responsible for enabling the construction of a
customizable theme for styling purposes. As long as the author (the
theme creator) enhances these capabilities via exposed variables, it
will manage CSS files that will provide styles to the pages and provide
functionalities to perform tweaks and changes to the theme. Therefore,
subsequent users of the same theme would be able to make certain
modifications to suit their needs, while retaining the already provided
set of opinionated styles. Priority: MEDIUM.

\paragraph{Stimulus/Response
sequences}\label{par:stimulusresponse-sequences-1}

Stimulus: Single or multiple files with preprocessed CSS attributes that
will receive the values of the exposed variables to be processed into a
final CSS file. Notice that the format of this file is not strictly CSS
as it should be noted that it is waiting for the building stage to add
the variables, therefore it's suggested to use a temporal file name
adding \texttt{.go.css} to differentiate it from its final version.

Response: Once processed, the file or set of files will be merged into a
single one, following an established structure and adding the exposed
variables via Go.

Stimulus: Authored parameters in the form of Go variables. There is also
the possibility of exposing such variables via YAML\cite{wikiyaml} or JSON\cite{wikijson} format, to
avoid overwriting Go code.

Response: The parameters are added to the final CSS file.

\textbf{REQ-1}: The exposed parameters should be added within the
preprocessed style file using the pattern
\texttt{\$\{{[}variable\ name{]}\}} in the same place where it will be
replaced.

\textbf{REQ-2}: These variables are then added or modified within a
specific Go module specifically designed for this purpose. Thereby, the
author will be able to expose these in this file, and add a default
value in case the user decides to not change anything. It should be
noted that the system won't take care of the compilation or verification
of the CSS file, hence it is completely up to the author to verify
whether it is correctly built.

\textbf{REQ-3}: Multiple style files could be added together in a single
final one, providing the author the flexibility to split them into
different modules for readability and maintainability purposes, thus
these must be specified in a Go module specifically designed for this
purpose, or within a main preprocessed style file using the nomenclature
\texttt{\$\{\#module:\ {[}file\ name{]}\}}.

\textbf{REQ-4}: The system must verify and prompt a clear error message
when there is a theme variable that has not been set. More specifically,
it should display a message using the following structure:
``\emph{Error: {[}variable name{]} has not been initialized.}'', where
the variable name will be replaced with the specific variable missing
its initialization.

\subsubsection{Templating}\label{subsubsec:templating}

\paragraph{Description and
priority}\label{par:description-and-priority-2}

The proposed system will incorporate a functionality that enables the
establishment of a predetermined framework or outline, in addition to
the primary template, for organizing the content into web pages. It is
noteworthy that the utilization of Go language's templating capabilities
is integral to the handling of this task. However, the system must
furnish a mechanism for retrieving the primary information components
from the markdown files. This will enable easy recognition and
referencing of said components in the template, thereby facilitating
customization. Priority: MEDIUM.

\paragraph{Stimulus/Response
sequences}\label{par:stimulusresponse-sequences-2}

Stimulus: Single or multiple HTML files with Go templating variables to
be filled with, from which the user can decide to use custom variables
(from custom Go code) or the main, already provided variables from the
markdown file components (headings, text, bold, italic, lists, etc).

Response: When built, a final HTML file should be prompted with the
provided content via markdown files and following the provided structure
or skeleton as in the template.

\paragraph{Functional requirements}\label{par:functional-requirements-1}

\textbf{REQ-1}: The processed format must follow the established
skeleton from the template. If there is no provided template, the system
must use a base one already provided with the generator.

\textbf{REQ-2}: Each markdown component should be accessible via Go
variables following this naming convention:

\begin{itemize}
    \item
    Heading level 1 (\texttt{\#}): \texttt{H1}
    \item
    Heading level 2 (\texttt{\#\#}): \texttt{H2}.
    \item
    Heading level 3 (\texttt{\#\#\#}): \texttt{H3}.
    \item
    Heading level 4 (\texttt{\#\#\#\#}): \texttt{H4}.
    \item
    Heading level 5 (\texttt{\#\#\#\#\#}): \texttt{H5}.
    \item
    Heading level 6 (\texttt{\#\#\#\#\#\#}): \texttt{H6}.
    \item
    Text: \texttt{B}.
    \item
    Bold text (\texttt{**text**}): \texttt{Strong}.
    \item
    Italic text (\texttt{\_text\_}): \texttt{Em}.
    \item
    Blockquote (\texttt{\textgreater{}}): \texttt{Blockquote}.
    \item
    Unordered list (\texttt{-}): \texttt{Ul}.
    \item
    Ordered list (\texttt{1.}, \texttt{2.}, \texttt{3.}, and so):
    \texttt{Ol}.
    \item
    Code (\texttt{\textasciigrave{}\textasciigrave{}}): \texttt{Code}.
    \item
    Fenced code block
    (\texttt{\textasciigrave{}\textasciigrave{}\textasciigrave{}\ \textasciigrave{}\textasciigrave{}\textasciigrave{}}):
    \texttt{Pre}.
    \item
    Link (\texttt{{[}text{]}(URL)}): \texttt{A}, \texttt{A.URL}.
    \item
    Image (\texttt{!{[}Alt\ text{]}(URL)}): \texttt{Img},
    \texttt{Img.Alt}, \texttt{Img.Src}.
\end{itemize}

It is important to note that the naming convention and HTML elements are
identical. This is intentional, as it should direct the user to the
correct element the component should be used in, so as to maintain best
practices for web accessibility and search engine optimization, avoiding
the use of meaningless \texttt{div} tags everywhere.

\subsubsection{Configuration files}\label{subsubsec:configuration-files}

\paragraph{Description and
priority}\label{par:description-and-priority-3}

Through the configuration files the user will be able to set up certain
parameters that will change the way the system works to adapt it to
given needs and setup, such as determine specific input and output
folders, theme name, template file name, author information (name,
website name, creation date), and others. It's worth noting that these
configuration files are not strictly necessary as the system must
initiate with a set of given default parameters, which can be
overwritten for extended customization. Priority: LOW.

\paragraph{Stimulus/Response
sequences}\label{par:stimulusresponse-sequences-3}

Stimulus: A Go\cite{donovan2015go}, JSON\cite{wikijson} or YAML\cite{wikiyaml} file with the set of parameters to be read
from, so the system can use them as variables for its configuration. The
type of file to be used will depend on the development time, as reading
directly from a Go file or module it's easier and faster to implement
than a JSON or YAML file, as it requires a parsing step first.

Response: The system will produce and/or build the static content
following these parameters accordingly.

\paragraph{Functional requirements}\label{par:functional-requirements-2}

\emph{REQ-1}: The system will use a set of default values if the
configuration files are not provided/changed, thus it is not strictly
needed for the system to work properly.

\emph{REQ-2}: The following configuration files must be scanned (found
via specific filename) and used for its respective purpose: Theme config
(\texttt{theme.*}, ) to specify styling parameters as well as the theme
name, template config (` layout.\emph{) to specify the files entries for
templating, main configuration (main.}) for the main parameters for
VaGo. File extension to be decided depending on development time as
noted on previous point.

\emph{REQ-3}: For each configuration file, the system must be able to
interpret and apply the following parameters:


1. Theme config (\texttt{theme.*}):


\begin{itemize}
    \item
    Theme name (\texttt{name}): The name of the given theme. This name
    will be accessed for further setup steps.
    \item
    Parameters(\texttt{params}): An open field to add a list of the open
    parameters for styling.
\end{itemize}

2. Template config (\texttt{layout.*}):

\begin{itemize}
    \item
    Entries (\texttt{entries}): A list of template filenames to be
    included in the templating system.
    \item
    Final layout (\texttt{final}): A nested ordered list of the templates
    to be merged for the final composition layout.
\end{itemize}

3. Main config (\texttt{main.*}):

\begin{itemize}
    \item
    Author name (\texttt{author}): The author name of the website to
    generated content for.
    \item
    Input folder (\texttt{input}): Input folder to get the files to read
    content from, prior to the website generation.
    \item
    Output folder (\texttt{output}): Output folder to write the static
    generation results.
    \item
    Theme (\texttt{theme}): The name of the theme to be used for this
    website.
\end{itemize}

It is worth noting that these configuration files and parameters are not
final, and yet other may get included in the future depending on the
needs that may arise.


    \setkeys{Gin}{width=1\linewidth}
    % Software design
    \markdownInput{chapters/software-design/design.md}

    % Used technologies
    \markdownInput{chapters/used-tech/technologies.md}

    \setkeys{Gin}{width=0.7\linewidth}
    % Implementation
    \markdownInput{chapters/implementation/implementation.md}

    % Results
    %! Author = javif
%! Date = 9/11/2023

% Preamble
\documentclass[11pt]{article}

% Packages
\usepackage{amsmath}

% Document
\begin{document}



\end{document}

    % Conclusions
    %! Author = javif
%! Date = 9/12/2023


\chapter{Conclusion}\label{ch:conclusion}

As demonstrated on previous chapter, VaGo is a software capable of obtaining input markdown files from an established
folder, read each file content, navigate through the abstract syntax tree, obtain the required tokens to build up a
web page, parse it into respective HTML tags, create HTML files accordingly to the input content using the same name,
build up dedicated styles using a theme system reading customization variables, read configuration file to adapt
system to user specifications, provide a listener server on a provided port, route requests to map the same file name
convention to target URLs, display insightful logging and an intuitive command interface with discerning usage
documentation.


As a result, it has been shown that the implementation of VaGo successfully meets the requirements outlined in
previous sections.
Furthermore, it can be regarded as a favorable alternative to existing Static Site Generators, as
it possesses the ability to execute a majority of the characteristics commonly observed in the present market.
This is without accounting for some other measures that are not the primary emphasis of this project, such as its
lightning-quick creation and serving of files, which opens up new possibilities beyond merely serving static
information: The rapid pace of building construction facilitates the expansion of serverside rendering capabilities.


What's more, VaGo has been developed using a minimalistic approach, reducing the code and the cognitive load to the
lowest possible, with a balance between fulfilling requirements and following the proposed design with the desired
simplicity.
This minimalistic philosophy has demonstrated an overall increased productivity in terms of implementation, while
reducing the amount of complications and challenges to face when dealing with bugs and errors, with the addition of
being an open door to extension given that the code base is very simple and easy to read and understand, allowing
future collaborators to improve its functioning and/or add new features.

All in all, VaGo has been capable of meeting the requirements established at the beginning, following the minimalistic
approach, providing a fully functional static site generator.
Hence, this project serves as a proof that selecting the correct tools, frameworks, language, libraries and modules,
by setting a realistic set of requirements, with the right approach that ties up the development process without
limiting its creativity, and understanding the balance between proposed design and implementation changes, any software
project can be developed with ease, ending up with a robust, flexible and open to extension results.

From the point of view of users, VaGo is ready to be used for those looking for a simple software that can allow them
to focus on content creation rather than software development details.
Create markdown content, input a couple of commands and have your static file served site ready to go.


    % Future lines
    %! Author = javif
%! Date = 9/12/2023

\chapter{Future lines}\label{ch:futurelines}

Although VaGo is capable of performing the task evaluated in this project, it is far from perfect, as there are some
improvements and features that can be added as a static site generator.

For instance, the software requires the whole code to be in the same parent folder as the input and output content,
styles, template, configuration and more.
Thus, it would be very handy to have a feature that allows the creation of a new VaGo project folder from scratch,
only with the required folders and files needed to start crafting content, without the code being in the same
directory, as it would allow the creation of multiple projects with the same code folder in another place in the
system, allowing the user to have a separation of concerns between the content and VaGo code base, useful for further
arrangement and manipulation.

On the other hand, given that it uses the concept of themes for styling, a theme package manager can be created along
with the main project to provide extended capabilities for theme publication and sharing, as well as download and
install other themes.
This will allow the user to follow the philosophy of only focusing on content creation and theme customization, as it
would subtract the need of crafting CSS styles files from scratch.

Finally, in order to meet the current state of the art of the most popular static site generators, VaGo must have the
capability of extending its functionalities via plugins, which must be easy to find, install, use and manage.
An example of a handy plugin, is the capability of adding LaTeX-like content writing, allowing the user to not only
create files on markdown format, but also extending these to use LaTex nomenclature, which is useful for
mathematicians as they can easily compose complex formulas.
This would require a plugin manager, to navigate through existing plugins and install them on command.

    % Bibliography
%    \markdownInput{chapters/bibliography.md}

    \bibliography{chapters/citations}
    \bibliographystyle{sapthesis}

\end{document}
