%! Author = javif
%! Date = 9/14/2023

\subsection{Specific requirements}\label{subsec:specific-requirements}

\subsubsection{External interfaces}\label{subsubsec:external-interfaces}

Users will be able to interact with the system using CLI commands to
carry out simple operations that are built into the system. Here, two
fundamental commands should be noted:

\begin{itemize}
    \item
    \texttt{Build}.
    \item
    \texttt{Serve}: To serve and share content via a specific port.
\end{itemize}

Additionally, the system ought to alert the user any time a command it
has issued contains an error because it is possible for them to
accidentally type the wrong command or misspell any of these. (using
buidl instead of build). Notably, unlike other systems, it will lack a
recommendation system to offer hints of similar words that are actually
interpreted as to assist the user in finding alternative commands, even
though it will inform the user about the existence of an error that
prevented the system from performing the required task.

\paragraph{Build}\label{par:build}

The task at hand involves the conversion of Markdown files into
web-based content, specifically in the form of HTML, JS, and CSS.

The program is designed to utilize the Command Line Interface (CLI) as
an input mechanism, whereby it recognizes the command name to execute
its designated task. Its output function involves generating the HTML
and CSS content into an \texttt{out} folder, in accordance with the
pre-existing Markdown files. It is noteworthy that each MD file ought to
generate an individual page to allow for the adaptability of
transforming every file into a distinct output for the website.

Throughout the translation process, it is imperative that the user is
kept informed of its progress. In the event of any errors, the system
should prompt error messages to ensure the accuracy of the MD files,
while simultaneously verifying that all necessary parameters for content
and styling have been provided. Upon completion, the system should
explicitly indicate the out folder path to the user.

Therefore, it should use the following format for each of the steps
considered for the translation and building process: {[}Date{]}{[}Page
Name{]}: Step, where each one of these detail token corresponds to:

\begin{itemize}
    \item
    Date: Specific date and time using the ISO 8601 format
    \href{https://en.wikipedia.org/wiki/ISO_8601}{1} without specifying
    the timezone as \texttt{VaGo} will limit itself to use the same
    timezone from the host machine. For instance: \emph{2023-04-09T14:
    22:05} should be used to represent the year 2023, month April (04),
    day 09 at 14 hours, 22 minutes and 05 seconds.
    \item
    Page Name: To specify the exact page name for the URL that takes to
    that specific page, which is the same as the MD file name.
    \item
    Step: To specify the building step it's currently on, whether it is
    identifying the MD tokens, performing translation to HTML and/or
    obtaining the parameters set for the theming and styling of the given
    page.
\end{itemize}

On the other hand, additional parameters can be added to the command for
specific output capabilities. For instance, \texttt{no-log} should be
using to omitting logging (no prompt message for every building process
step) except for error and finalization message. Also, \texttt{no-time}
should be used to omit the timestamp from the steps, prompting a message
with the following structure: \emph{{[}Page name{]}:Step}, with the
purpose of avoiding overloading logs and make them easier to read.

\paragraph{Serve}\label{par:serve}

The second command serves the purpose of keep the system continuously
executing, hence no accepting other commands in the meantime, in order
to open a port and send the already built files (by means of the
previous command, \texttt{build}) via HTTP.

It is important to note that this command will exclusively serve content
available in the \texttt{out} folder, therefore if the previous command
hasn't been run, or if there are no files in this folder, then it won't
be able to serve any content at all. Because of it, it is required that
the user must first build content in order to then serve it.

Moreover, in addition to the behavior noted above, it must output logs
to inform the user of the current status of the serving process,
starting by indicating as soon as it starts reading content and
initiating listening for requests, while also informing of any issue
that may occur by prompting error messages.

Once it starts listening for requests, it should inform of any incoming
request/response using the following format: {[}Date{]}: Sending
        {[}Page{]} to {[}requester IP{]}, where each one of these token
corresponds to:

\begin{itemize}
    \item
    Date: Same as the one outlined for \texttt{build} command, using the
    ISO 8601 format without timezone.
    \item
    Page: Page name as stated in the MD filename.
    \item
    Requester IP: Specific IP address of the incoming IP.
\end{itemize}

Additionally, an extra (optional) parameter can be provided to outline
the specific port the system should open to listen for requests, instead
of the default one.

\subsubsection{Functions}\label{subsubsec:functions}

The following provides a more in-depth explanation of the functional
areas, including full requirements.

\paragraph{Generate page}\label{par:generate-page}

\subparagraph{Description and priority}\label{subpar:description-and-priority}

The task at hand involves generating a page document through the process
of translating the contents of a Markdown file into HTML. This requires
the utilization of appropriate tags to accurately map the entities
present within each Markdown block. Priority: HIGH.

\subparagraph{Stimulus/Response
sequences}\label{subpar:stimulusresponse-sequences}

Stimulus: This feature must obtain the markdown files from a
\texttt{source} folder.

Response: Translate or generate a corresponding HTML file for each one
of these MD files, outputting the result pages into an \texttt{out}
folder.

\subparagraph{Functional requirements}\label{subpar:functional-requirements}

\textbf{REQ-1}: The MD blocks must be interpreted and translated to HTML
tags to generate a page accordingly to what the user has described
initially. It should, at least, implement the following elements to
ensure a fully accessible page as per best practices for web standards:

\begin{itemize}
    \item
    Heading level 1 (\texttt{\#}): Heading 1
    (\texttt{\textless{}h1\textgreater{}\textless{}/h1\textgreater{}}).
    \item
    Heading level 2 (\texttt{\#\#}): Heading 2
    (\texttt{\textless{}h2\textgreater{}\textless{}/h2\textgreater{}}).
    \item
    Heading level 3 (\texttt{\#\#\#}): Heading 3
    (\texttt{\textless{}h3\textgreater{}\textless{}/h3\textgreater{}}).
    \item
    Heading level 4 (\texttt{\#\#\#\#}): Heading 4
    (\texttt{\textless{}h4\textgreater{}\textless{}/h4\textgreater{}}).
    \item
    Heading level 5 (\texttt{\#\#\#\#\#}): Heading 5
    (\texttt{\textless{}h5\textgreater{}\textless{}/h5\textgreater{}}).
    \item
    Heading level 6 (\texttt{\#\#\#\#\#\#}): Heading 6
    (\texttt{\textless{}h6\textgreater{}\textless{}/h6\textgreater{}}).
    \item
    Text: Paragraph
    (\texttt{\textless{}p\textgreater{}\textless{}/p\textgreater{}}).
    \item
    Bold text (\texttt{**text**}): Strong
    (\texttt{\textless{}strong\textgreater{}\textless{}/strong\textgreater{}}).
    \item
    Italic text (\texttt{\_text\_}): Emphasis
    (\texttt{\textless{}em\textgreater{}\textless{}/em\textgreater{}}).
    \item
    Blockquote (\texttt{\textgreater{}}): Blockquote
    (\texttt{\textless{}blockquote\textgreater{}\textless{}/blockquote\textgreater{}}).
    \item
    Unordered list (\texttt{-}): Unordered list
    (\texttt{\textless{}ul\textgreater{}\textless{}/ul\textgreater{}})
    surrounding list items
    (\texttt{\textless{}li\textgreater{}\textless{}/li\textgreater{}}).
    \item
    Ordered list (\texttt{1.}, \texttt{2.}, \texttt{3.}, and so): Ordered
    list
    (\texttt{\textless{}ol\textgreater{}\textless{}/ol\textgreater{}})
    surrounding list items
    (\texttt{\textless{}li\textgreater{}\textless{}/li\textgreater{}}).
    \item
    Code (\texttt{\textasciigrave{}\textasciigrave{}}): Code
    (\texttt{\textless{}code\textgreater{}\textless{}/code\textgreater{}}).
    \item
    Fenced code block
    (\texttt{\textasciigrave{}\textasciigrave{}\textasciigrave{}\ \textasciigrave{}\textasciigrave{}\textasciigrave{}}):
    Preformatted text
    (\texttt{\textless{}pre\textgreater{}\textless{}/pre\textgreater{}})
    surrounding a code block
    (\texttt{\textless{}code\textgreater{}\textless{}/code\textgreater{}}).
    \item
    Link (\texttt{{[}text{]}(URL)}): Anchor tag along with the hyperlink
    attribute
    (\texttt{\textless{}a\ href="URL"\textgreater{}text\textless{}/a\textgreater{}}).
    \item
    Image (\texttt{!{[}Alt\ text{]}(URL)}): Image tag along with alt text
    and source attributes
    (\texttt{\textless{}img\ alt="Alt\ text"\ src="URL"\textgreater{}\textless{}/img\textgreater{}}).
\end{itemize}

\textbf{REQ-2}: It must verify whenever there is an input error or
incorrect usage of the MD blocks, and thereby let the user know about
these via error messages and halt the translation process immediately.

\textbf{REQ-3}: Each generated page will have the same file name as the
input MD file.

\paragraph{Theme styling creation and
customization}\label{par:theme-styling-creation-and-customization}

\subparagraph{Description and
priority}\label{subpar:description-and-priority-1}

This feature is responsible for enabling the construction of a
customizable theme for styling purposes. As long as the author (the
theme creator) enhances these capabilities via exposed variables, it
will manage CSS files that will provide styles to the pages and provide
functionalities to perform tweaks and changes to the theme. Therefore,
subsequent users of the same theme would be able to make certain
modifications to suit their needs, while retaining the already provided
set of opinionated styles. Priority: MEDIUM.

\subparagraph{Stimulus/Response
sequences}\label{subpar:stimulusresponse-sequences-1}

Stimulus: Single or multiple files with preprocessed CSS attributes that
will receive the values of the exposed variables to be processed into a
final CSS file. Notice that the format of this file is not strictly CSS
as it should be noted that it is waiting for the building stage to add
the variables, therefore it's suggested to use a temporal file name
adding \texttt{.go.css} to differentiate it from its final version.

Response: Once processed, the file or set of files will be merged into a
single one, following an established structure and adding the exposed
variables via Go.

Stimulus: Authored parameters in the form of Go variables. There is also
the possibility of exposing such variables via YAML or JSON format, to
avoid overwriting Go code.

Response: The parameters are added to the final CSS file.

\textbf{REQ-1}: The exposed parameters should be added within the
preprocessed style file using the pattern
\texttt{\$\{{[}variable\ name{]}\}} in the same place where it will be
replaced.

\textbf{REQ-2}: These variables are then added or modified within a
specific Go module specifically designed for this purpose. Thereby, the
author will be able to expose these in this file, and add a default
value in case the user decides to not change anything. It should be
noted that the system won't take care of the compilation or verification
of the CSS file, hence it is completely up to the author to verify
whether it is correctly built.

\textbf{REQ-3}: Multiple style files could be added together in a single
final one, providing the author the flexibility to split them into
different modules for readability and maintainability purposes, thus
these must be specified in a Go module specifically designed for this
purpose, or within a main preprocessed style file using the nomenclature
\texttt{\$\{\#module:\ {[}file\ name{]}\}}.

\textbf{REQ-4}: The system must verify and prompt a clear error message
when there is a theme variable that has not been set. More specifically,
it should display a message using the following structure:
``\emph{Error: {[}variable name{]} has not been initialized.}'', where
the variable name will be replaced with the specific variable missing
its initialization.

\paragraph{Templating}\label{par:templating}

\subparagraph{Description and
priority}\label{subpar:description-and-priority-2}

The proposed system will incorporate a functionality that enables the
establishment of a predetermined framework or outline, in addition to
the primary template, for organizing the content into web pages. It is
noteworthy that the utilization of Go language's templating capabilities
is integral to the handling of this task. However, the system must
furnish a mechanism for retrieving the primary information components
from the markdown files. This will enable easy recognition and
referencing of said components in the template, thereby facilitating
customization. Priority: MEDIUM.

\subparagraph{Stimulus/Response
sequences}\label{subpar:stimulusresponse-sequences-2}

Stimulus: Single or multiple HTML files with Go templating variables to
be filled with, from which the user can decide to use custom variables
(from custom Go code) or the main, already provided variables from the
markdown file components (headings, text, bold, italic, lists, etc).

Response: When built, a final HTML file should be prompted with the
provided content via markdown files and following the provided structure
or skeleton as in the template.

\subparagraph{Functional requirements}\label{subpar:functional-requirements-1}

\textbf{REQ-1}: The processed format must follow the established
skeleton from the template. If there is no provided template, the system
must use a base one already provided with the generator.

\textbf{REQ-2}: Each markdown component should be accessible via Go
variables following this naming convention:

\begin{itemize}
    \item
    Heading level 1 (\texttt{\#}): \texttt{H1}
    \item
    Heading level 2 (\texttt{\#\#}): \texttt{H2}.
    \item
    Heading level 3 (\texttt{\#\#\#}): \texttt{H3}.
    \item
    Heading level 4 (\texttt{\#\#\#\#}): \texttt{H4}.
    \item
    Heading level 5 (\texttt{\#\#\#\#\#}): \texttt{H5}.
    \item
    Heading level 6 (\texttt{\#\#\#\#\#\#}): \texttt{H6}.
    \item
    Text: \texttt{B}.
    \item
    Bold text (\texttt{**text**}): \texttt{Strong}.
    \item
    Italic text (\texttt{\_text\_}): \texttt{Em}.
    \item
    Blockquote (\texttt{\textgreater{}}): \texttt{Blockquote}.
    \item
    Unordered list (\texttt{-}): \texttt{Ul}.
    \item
    Ordered list (\texttt{1.}, \texttt{2.}, \texttt{3.}, and so):
    \texttt{Ol}.
    \item
    Code (\texttt{\textasciigrave{}\textasciigrave{}}): \texttt{Code}.
    \item
    Fenced code block
    (\texttt{\textasciigrave{}\textasciigrave{}\textasciigrave{}\ \textasciigrave{}\textasciigrave{}\textasciigrave{}}):
    \texttt{Pre}.
    \item
    Link (\texttt{{[}text{]}(URL)}): \texttt{A}, \texttt{A.URL}.
    \item
    Image (\texttt{!{[}Alt\ text{]}(URL)}): \texttt{Img},
    \texttt{Img.Alt}, \texttt{Img.Src}.
\end{itemize}

It is important to note that the naming convention and HTML elements are
identical. This is intentional, as it should direct the user to the
correct element the component should be used in, so as to maintain best
practices for web accessibility and search engine optimization, avoiding
the use of meaningless \texttt{div} tags everywhere.

\paragraph{Configuration files}\label{par:configuration-files}

\subparagraph{Description and
priority}\label{subpar:description-and-priority-3}

Through the configuration files the user will be able to set up certain
parameters that will change the way the system works to adapt it to
given needs and setup, such as determine specific input and output
folders, theme name, template file name, author information (name,
website name, creation date), and others. It's worth noting that these
configuration files are not strictly necessary as the system must
initiate with a set of given default parameters, which can be
overwritten for extended customization. Priority: LOW.

\subparagraph{Stimulus/Response
sequences}\label{subpar:stimulusresponse-sequences-3}

Stimulus: A Go, JSON or YAML file with the set of parameters to be read
from, so the system can use them as variables for its configuration. The
type of file to be used will depend on the development time, as reading
directly from a Go file or module it's easier and faster to implement
than a JSON or YAML file, as it requires a parsing step first.

Response: The system will produce and/or build the static content
following these parameters accordingly.

\subparagraph{Functional requirements}\label{subpar:functional-requirements-2}

\emph{REQ-1}: The system will use a set of default values if the
configuration files are not provided/changed, thus it is not strictly
needed for the system to work properly.

\emph{REQ-2}: The following configuration files must be scanned (found
via specific filename) and used for its respective purpose: Theme config
(\texttt{theme.*}, ) to specify styling parameters as well as the theme
name, template config (` layout.\emph{) to specify the files entries for
templating, main configuration (main.}) for the main parameters for
VaGo. File extension to be decided depending on development time as
noted on previous point.

\emph{REQ-3}: For each configuration file, the system must be able to
interpret and apply the following parameters:


1. Theme config (\texttt{theme.*}):


\begin{itemize}
    \item
    Theme name (\texttt{name}): The name of the given theme. This name
    will be accessed for further setup steps.
    \item
    Parameters(\texttt{params}): An open field to add a list of the open
    parameters for styling.
\end{itemize}

2. Template config (\texttt{layout.*}):

\begin{itemize}
    \item
    Entries (\texttt{entries}): A list of template filenames to be
    included in the templating system.
    \item
    Final layout (\texttt{final}): A nested ordered list of the templates
    to be merged for the final composition layout.
\end{itemize}

3. Main config (\texttt{main.*}):

\begin{itemize}
    \item
    Author name (\texttt{author}): The author name of the website to
    generated content for.
    \item
    Input folder (\texttt{input}): Input folder to get the files to read
    content from, prior to the website generation.
    \item
    Output folder (\texttt{output}): Output folder to write the static
    generation results.
    \item
    Theme (\texttt{theme}): The name of the theme to be used for this
    website.
\end{itemize}

It is worth noting that these configuration files and parameters are not
final, and yet other may get included in the future depending on the
needs that may arise.
